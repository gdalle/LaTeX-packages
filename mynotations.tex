\renewcommand{\epsilon}{\varepsilon}
\newcommand{\Ee}{\mathds{E}}
\newcommand{\Pp}{\mathds{P}}
\newcommand{\Vv}{\mathds{V}}
\newcommand{\Rr}{\mathds{R}}
\newcommand{\Nn}{\mathds{R}}
\newcommand{\Cov}{\mathrm{Cov}}
\newcommand{\Tr}{\mathrm{Tr}}
\newcommand{\diag}{\mathrm{diag}}
\newcommand{\one}{\mathds{1}}
\newcommand{\KL}[2]{\mathrm{KL}\left(#1 ~\middle\Vert~ #2 \right)}
\newcommand{\e}{\mathbf{e}}
\newcommand{\Idmat}{\mathrm{I}}

\renewcommand\qedsymbol{$\blacksquare$}

\newtheorem{theorem}{Theorem}[section]
\newtheorem{corollary}{Corollary}[section]
\newtheorem{lemma}{Lemma}[section]
\newtheorem{definition}{Definition}[section]
\newtheorem{remark}{Remark}


\newenvironment{todo}{
 \begin{mdframed}[
   roundcorner=10,
   nobreak=true,
   frametitle={TO DO},
   frametitlerule=true,
   frametitlebackgroundcolor=red!20,
   backgroundcolor=red!10,
   font={\itshape},
  ]
  }{
 \end{mdframed}
}

\newenvironment{rem}{
 \begin{mdframed}[
   roundcorner=10,
   nobreak=true,
   frametitle={REMARK},
   frametitlerule=true,
   frametitlebackgroundcolor=blue!20,
   backgroundcolor=blue!10,
   font={\itshape},
  ]
  }{
 \end{mdframed}
}